\documentclass[sotsuron]{jcsie}
\title{ライブストリーミングに対応した分散ハッシュテーブルの検討}
\etitle{Title}
\author{沈 嘉秋}
\eauthor{Yoshiaki Shin}
\id{T19I917F}
\keywords{ほげ, ほげ}
\ekeywords{hoge, hoge}
\begin{document}
\maketitle
\emaketitle
\pagenumbering{roman}
\begin{abstract}    
  分散ハッシュテーブルはファイル共有サービス等に応用され、
  すでに普及している一方で、
  近年はブロックチェーンの関連技術としても注目を集めている。
  分散ハッシュテーブルの中でもKademliaはその実装の容易さと
  ノードの出入りに対する耐性の強さから
  実用的なサービスへの応用が可能であり、
  多くのサービスで採用されている。
  近年、ライブストリーミングサービスの需要が高まっている
  一方で配信プラットフォームの事業者への依存が強く
  サービス利用者の立場は弱いものとなっている。
  そこで本論文では
  分散的なライブストリーミングサービスの基盤となる
  システムをKademliaを元に実装し、その評価を行った。
  Kademlia上で単純にライブストリーミングの実装を行う場合、
  非効率的な通信が発生するが、本論文の手法では、
  Kademliaネットワークを更に構造化することで
  非効率的な通信を削減することが可能となった。
  結果、分散的なライブストリーミングサービスの実装への
  足がかりを作ることが出来た。
\end{abstract}
\begin{eabstract}
  Here is abstract
  Here is abstract
  Here is abstract
  Here is abstract
  Here is abstract
  Here is abstract
  Here is abstract
  Here is abstract
  Here is abstract
  Here is abstract
  Here is abstract
  Here is abstract
  Here is abstract
  Here is abstract
\end{eabstract}
\tableofcontents
\pagenumbering{arabic}
\chapter{はじめに}
\section{背景}
\section{目的}
\chapter{知識}
\begin{acknowledgment}
\end{acknowledgment}
\nocite{*}
\bibliographystyle{plain}
\bibliography{reference}
\end{document}
